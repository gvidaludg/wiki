\documentclass{article}
\usepackage[T1]{fontenc}
\usepackage[catalan]{babel}
\usepackage{listings}

\lstset{basicstyle=\ttfamily}

\begin{document}

\title{Similitud entre documents (Part II)}
\author{Guillem Vidal}
\maketitle

\section{Introducció}

M'ha semblat adient modificar el \lstinline{MapReduce} per tal d'optimitzar més
les operacions, i fer-lo més genèric de manera que pogués acceptar més diversitat
de tipus.

La funció que genera tot el funcionament amb actors es diu \lstinline{groupMapReduce},
i té la següent signatura:

\begin{lstlisting}
def groupMapReduce[K, A, B, C](
	input: Iterable[A],
	compute: A => Iterable[B],
	key: B => K,
	mapper: B => C,
	reducer: (C, C) => C,
	nmappers: Int = 16,
	nreducers: Int = 16
): Map[K, C]
\end{lstlisting}

El paràmetre \lstinline{compute} és redundant: permet que durant el mapeig es
generi una sèrie de valors a agrupar i reduir, que no sigui una relació 1 a 1
amb l'input, sinó 1 a N (opcionalment).

Això és útil quan hem de calcular les aparicions de les paraules en els
documents, per exemple, ja que podem passar com a input els documents i generar
totes les combinacions a dins del \lstinline{groupMapReduce} aprofitant les
seves aptituds de paral·lelisme.

\section{Demostracions}

\subsection{La funció \lstinline{timeMeasurement}}

L'he definida de la següent manera:

\begin{lstlisting}
def timeMeasurement[A](
	compute: Unit => A,
	name: String = "Function"
): A = {

	val beg = System.currentTimeMillis()

	val ret = compute()

	val end = System.currentTimeMillis()

	val elapsed = (end - beg).toFloat / 1000;
	println(s"$name took $elapsed seconds!")

	return ret
}
\end{lstlisting}

No he volgut utilitzar la sintaxi especial per l'avaluació \emph{lazy} de les
expressions, perquè m'agrada la claredat que el que s'està enviant és una
expressió, no el valor d'aquesta.

\subsection{Nombre promig de referències}

\begin{verbatim}
112 references on average.
\end{verbatim}

\subsection{\emph{Query} de recomanació mitjançant PR}

Si busquem per 'Guerra', aquests són els primers resultats:

\begin{verbatim}
Natal (Rio Grande do Norte),
Heinkel He 280,
Fokker F.VII,
Raymond Collishaw,
Yokosuka MXY-7,
Diàspora basca,
Països àrabs,
Edat Contemporània als Països Catalans,
Història de Bielorússia,
...
\end{verbatim}

\subsection{Pàgines que s'assemblen sense referenciar-se}

Donada una \emph{query} de 'Guerra' i agafant els 100 primers elements retornats,
obtinc els següents resultats:

\begin{verbatim}
(Economia de la Unió Soviètica, Èxode rural),
(Història d'Estònia, Èxode rural),
(Persona desplaçada, Èxode rural),
(Feixisme italià, Èxode rural),
(Història d'Amèrica, Èxode rural),
(Llista de diàspores, Èxode rural),
...
\end{verbatim}

\subsection{Eficiència}

\begin{tabular}{|c|c|c|c|c|}
	\hline
	\textbf{Mappers} & \textbf{Reducers} & \textbf{Load} & \textbf{Query} & \textbf{Similar} \\
	\hline
	1 & 1 & 15.437 & 0.61 & 1.677 \\
	2 & 2 & 8.594 & 0.615 & 2.23 \\
	4 & 4 & 8.667 & 0.957 & 3.873 \\
	8 & 8 & 10.37 & 1.067 & 2.286 \\
	\hline
\end{tabular}

\end{document}
